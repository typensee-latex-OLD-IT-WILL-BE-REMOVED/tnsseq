\documentclass[12pt,a4paper]{article}

\makeatletter
	\input{../config/header[fr].sty}

	\usepackage{01-big-small-O-n-co}
\makeatother


% == EXTRAS == %



\begin{document}

% \section{Analysis}

\subsection{Comparaison asymptotique de suites et de fonctions}

\subsubsection{\texorpdfstring{Les notations $\bigO{}$ et $\smallO{}$}%
                              {Les notations "grand O" et "petit O"}}

\newparaexample{}

Les notations suivantes sont dues à Landau.

\begin{latexex}
$\bigO{}$ ou $\smallO{}$
\end{latexex}


% ---------------------- %


\newparaexample{}

\begin{latexex}
$\bigO{x} \neq \smallO{x}$ ou
$e^{t + \smallO{t}} = e^{\bigO{t}}$
\end{latexex}


% ---------------------- %


\subsubsection{Fiches techniques}

\paragraph{\texorpdfstring{Les notations $\bigO{}$ et $\smallO{}$}%
                          {Les notations "grand O" et "petit O"}}

\IDmacro*{bigO}{1}

\IDmacro*{smallO}{1}

\IDarg{} si l'argument est vide, il est ignoré, sinon il est mis entre des parenthèses après $\bigO{}$ ou $\smallO{}$.


% ---------------------- %


\subsubsection{\texorpdfstring{La notation $\bigomega{}$}%
                              {La notation "grand Omega"}}

\newparaexample{}

La notation suivante est due à Hardy et Littlewood.

\begin{latexex}
$\bigomega{}$
\end{latexex}


% ---------------------- %


\newparaexample{}

Dans l'exemple suivant, $f(n) = \bigomega{g(n)}$ signifie :
$\exists (m, n_0)$ tel que $n \geq n_0$ implique $f(n) \geq m g(n)$.

\begin{latexex}
$f(n) = \bigomega{g(n)}$
\end{latexex}


% ---------------------- %


\subsubsection{Fiches techniques}

\paragraph{\texorpdfstring{La notation $\bigomega{}$}%
                          {La notation "grand Omega"}}

\IDmacro*{bigomega}{1}

\IDarg{} si l'argument est vide, il est ignoré, sinon il est mis entre des parenthèses après $\bigomega{}$.


% ---------------------- %


\subsubsection{\texorpdfstring{La notation $\bigtheta{}$}%
                              {La notation "grand Theta"}}

\newparaexample{}

\begin{latexex}
$\bigtheta{}$
\end{latexex}


% ---------------------- %


\newparaexample{}

Dans l'exemple suivant, $f(n) = \bigtheta{g(n)}$ signifie : $\exists (m, M, n_0)$ tel que $m g(n) \leq f(n) \leq M g(n)$ dès que $n \geq n_0$.

\begin{latexex}
$f(n) = \bigtheta{g(n)}$
\end{latexex}


% ---------------------- %


\subsubsection{Fiches techniques}

\paragraph{\texorpdfstring{La notation $\bigtheta{}$}%
                          {La notation "grand Theta"}}

\IDmacro*{bigtheta}{1}

\IDarg{} si l'argument est vide, il est ignoré, sinon il est mis entre des parenthèses après $\bigtheta{}$.

\end{document}
