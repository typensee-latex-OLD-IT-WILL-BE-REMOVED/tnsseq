%\documentclass[12pt,a4paper]{scrartcl}
\documentclass[12pt,a4paper]{article}

\makeatletter % Technical doc - START

% ---------------------- %
% -- GENERAL SETTINGS -- %
% ---------------------- %

\usepackage[
	top    = 2cm,
	bottom = 2cm,
	left   = 1.5cm,
	right  = 1.5cm
]{geometry}

\usepackage[utf8]{inputenc}
\usepackage[T1]{fontenc}
\usepackage{ucs}

\usepackage[french]{babel,varioref}

\usepackage{color}
\usepackage{hyperref}
\hypersetup{
    colorlinks,
    citecolor = black,
    filecolor = black,
    linkcolor = black,
    urlcolor  = black
}
\usepackage[numbered]{bookmark}

\usepackage{enumitem}
\usepackage{multicol}
\usepackage{longtable}
\usepackage{makecell}

\setlength{\parindent}{0cm}
\setlist{noitemsep}



% --------------- %
% -- TOC & Co. -- %
% --------------- %

\usepackage[raggedright]{titlesec}

%\renewcommand\thechapter{\Alph{chapter}.}
\renewcommand\thesection{\Roman{section}.}
\renewcommand\thesubsection{\arabic{subsection}.}
\renewcommand\thesubsubsection{\roman{subsubsection}.}


\titleformat{\paragraph}[hang]{\normalfont\normalsize\bfseries}{\theparagraph}{1em}{}
\titlespacing*{\paragraph}{0pt}{3.25ex plus 1ex minus .2ex}{0.5em}


% Source
%    * https://tex.stackexchange.com/a/558025/6880
\usepackage{tocbasic}[2020/07/22]% needs KOMA-Script version 3.31

\DeclareTOCStyleEntries[
    raggedentrytext,
    linefill = \hfill,
    indent   = 0pt,
    dynindent,
    numwidth = 0pt,
    numsep   = 1ex,
    dynnumwidth
]{tocline}{
	chapter,
	section,
	subsection,
	subsubsection,
	paragraph,
	subparagraph
}

\DeclareTOCStyleEntry[indentfollows = chapter]{tocline}{section}



% ----------- %
% -- TOOLS -- %
% ----------- %

\usepackage{ifplatform}
\usepackage{ifthen}
\usepackage{macroenvsign}
\usepackage{pgffor}



% ------------------------- %
% -- SPECIAL FORMATTINGS -- %
% ------------------------- %

\usepackage{amsthm}

\usepackage{tcolorbox}


% -- LISTINGS -- %

%\tcbuselibrary{listingsutf8}
\tcbuselibrary{minted, breakable}

\newtcblisting{latexex}{%
    breakable,
    sharp corners,%
    left   = 1mm, right = 1mm,%
    bottom = 1mm, top   = 1mm,%
    %colupper = red!75!blue,%
    listing side text
}

\newtcbinputlisting{\inputlatexex}[2][]{%
    listing file={#2},%
    breakable,
    sharp corners,%
    left   = 1mm, right = 1mm,%
    bottom = 1mm, top   = 1mm,%
    %colupper = red!75!blue,%
    listing side text
}


\newtcblisting{latexex-flat}{%
    breakable,
    sharp corners,%
    left   = 1mm, right = 1mm,%
    bottom = 1mm, top   = 1mm,%
    %colupper = red!75!blue,%
}

\newtcbinputlisting{\inputlatexexflat}[2][]{%
    listing file={#2},%
    breakable,
    sharp corners,%
    left   = 1mm, right = 1mm,%
    bottom = 1mm, top   = 1mm,%
    %colupper = red!75!blue,%
}


\newtcblisting{latexex-alone}{%
    breakable,
    sharp corners,%
    left   = 1mm, right = 1mm,%
    bottom = 1mm, top   = 1mm,%
    %colupper = red!75!blue,%
    listing only
}

\newtcbinputlisting{\inputlatexexalone}[2][]{%
    listing file={#2},%
    breakable,
    sharp corners,%
    left   = 1mm, right = 1mm,%
    bottom = 1mm, top   = 1mm,%
    %colupper = red!75!blue,%
    listing only
}


\newcommand\inputlatexexcodeafter[1]{%
    \begin{center}
        \input{#1}
    \end{center}

    \vspace{-.5em}
    
    Le rendu précédent a été obtenu via le code suivant.
    
    \inputlatexexalone{#1}
}


\newcommand\inputlatexexcodebefore[1]{%
    \inputlatexexalone{#1}
    \vspace{-.75em}
    \begin{center}
        \textit{\footnotesize Rendu du code précédent}
        
        \medskip
        
        \input{#1}
    \end{center}
}


% -- REMARK -- %

\theoremstyle{definition}
\newtheorem*{remark}{Remarque}


% -- EXAMPLE -- %

\newcounter{paraexample}[subsubsection]

\newcommand\@newexample@abstract[2]{%
    \paragraph{%
        #1%
        \if\relax\detokenize{#2}\relax\else {} -- #2\fi%
    }%
}

\newcommand\newparaexample{\@ifstar{\@newparaexample@star}{\@newparaexample@no@star}}

\newcommand\@newparaexample@no@star[1]{%
    \refstepcounter{paraexample}%
    \@newexample@abstract{Exemple \theparaexample}{#1}%
}

\newcommand\@newparaexample@star[1]{%
    \@newexample@abstract{Exemple}{#1}%
}


% -- CHANGE LOG -- %

\newcommand\topic{\@ifstar{\@topic@star}{\@topic@no@star}}

\newcommand\@topic@no@star[1]{%
    \textbf{\textsc{#1}.}%
}

\newcommand\@topic@star[1]{%
    \textbf{\textsc{#1} :}%
}


% -- ABOUT MACROS & Co. -- %

\newcommand\env[1]{\texttt{#1}}
\newcommand\macro[1]{\env{\textbackslash{}#1}}

\newcommand\separation{
    \medskip
    \hfill\rule{0.5\textwidth}{0.75pt}\hfill
    \medskip
}


\newcommand\extraspace{
    \vspace{0.25em}
}


\newcommand\whyprefix[2]{%
    \textbf{\prefix{#1}}-#2%
}

\newcommand\mwhyprefix[2]{%
    \texttt{#1 = #1-#2}%
}

\newcommand\prefix[1]{%
    \texttt{#1}%
}


\newcommand\inenglish{\@ifstar{\@inenglish@star}{\@inenglish@no@star}}

\newcommand\@inenglish@star[1]{%
    \emph{\og #1 \fg}%
}

\newcommand\@inenglish@no@star[1]{%
    \@inenglish@star{#1} en anglais%
}


\newcommand\ascii{\texttt{ASCII}}

\makeatother % Technical doc - END


\usepackage{tnsseq}


\begin{document}

\renewcommand\labelitemi{\raisebox{0.125em}{\tiny\textbullet}}
\renewcommand{\labelitemii}{---}

\title{  %
	Le package \texttt{tnsseq}:\\%
	théorie générale des suites\\%
	{\footnotesize Code source disponible sur \url{https://github.com/typensee-latex/tnsseq.git}.}\\%
{\footnotesize Version \texttt{0.2.0-beta} développée et testée sur \macosxname{}.}%
}
\author{Christophe BAL}
\date{2021-03-03}

\maketitle


\vspace{2em}

\hrule

\tableofcontents

\vspace{1.5em}

\hrule

\newpage

\section{Introduction}

Le package \verb+tnsseq+ propose quelques macros utiles quand l'on parle de suites ou de séries. La saisie proposée se veut sémantique et simple.


% tnscom used - START
\section{Beta-dépendance}

\verb#\tnscom# qui est disponible sur \url{https://github.com/typensee-latex/tnscom.git} est un package utilisé en coulisse.
% tnscom used - END
% List of packages - START
\section{Packages utilisés}

La roue ayant déjà été inventée, le package \verb#tnsseq# utilise les packages suivants sans aucun scrupule.

\begin{multicols}{4}
    \begin{itemize}
        \item \verb#amssymb#
        \item \verb#mathtools#
    \end{itemize}
\end{multicols}
% List of packages - END
\section{Des notations complémentaires pour des suites spéciales}

Voici trois types de suites avec deux ou quatre indices.

\begin{latexex}
$\seqplus{F}{1}{2}$

$\seqhypergeo{F}{1}{2}$

$\seqsuprageo{F}{1}{2}{3}{4}$
pour les fous\dots :-)
\end{latexex}


% ---------------------- %
\section{Sommes et produits en mode ligne}

Pour limiter l'espace, \LaTeX{} affiche $\sum_{k=0}^{n}$ et non $\dsum_{k=0}^{n}$ sauf si l'on utilise la commande \macro{displaystyle}.
Les macros \macro{dsum} et \macro{dprod} permettent de se passer de \macro{displaystyle}.
Voici un exemple.


\begin{latexex}
 $\dsum_{k=0}^{n} 2^k
= \sum_{k=0}^{n} 2^k$

 $\dprod_{k=1}^{n} k
= \prod_{k=1}^{n} k$
\end{latexex}


\begin{remark}
	On peut taper  $\dsum_{k=0}^{n} \frac{1}{k}$ où la fraction n'est pas en mode \macro{displaystyle}.
\end{remark}


% ---------------------- %
\section{Comparaison asymptotique de suites et de fonctions}

\subsection{\texorpdfstring{Les notations $\bigO$ et $\smallO$}%                           {Les notations "grand O" et "petit O"}}
                           {Les notations "grand O" et "petit O"}}

\newparaexample{}

Les notations suivantes sont dues à Landau.

\begin{latexex}
$\bigO$ ou $\smallO$
\end{latexex}


% ---------------------- %


\newparaexample{}

\begin{latexex}
$\bigO(x) \neq \smallO(x)$ ou
$e^{t + \smallO(t)} = e^{\bigO(t)}$
\end{latexex}


% ---------------------- %


\subsection{\texorpdfstring{La notation $\bigomega$}%                           {La notation "grand Omega"}}
                           {La notation "grand Omega"}}

\newparaexample{}

La notation suivante est due à Hardy et Littlewood.

\begin{latexex}
$\bigomega$
\end{latexex}


% ---------------------- %


\newparaexample{}

Dans l'exemple suivant, $f(n) = \bigomega(g(n))$ signifie :
$\exists (m, n_0)$ tel que $n \geq n_0$ implique $f(n) \geq m g(n)$.

\begin{latexex}
$f(n) = \bigomega(g(n))$
\end{latexex}


% ---------------------- %


\subsection{\texorpdfstring{La notation $\bigtheta$}%                           {La notation "grand Theta"}}
                           {La notation "grand Theta"}}

\newparaexample{}

\begin{latexex}
$\bigtheta$
\end{latexex}


% ---------------------- %


\newparaexample{}

Dans l'exemple suivant, $f(n) = \bigtheta(g(n))$ signifie : $\exists (m, M, n_0)$ tel que $m g(n) \leq f(n) \leq M g(n)$ dès que $n \geq n_0$.

\begin{latexex}
$f(n) = \bigtheta(g(n))$
\end{latexex}


% ---------------------- %
\newpage

\section{Historique}

Nous ne donnons ici qu'un très bref historique récent
\footnote{
	On ne va pas au-delà de un an depuis la dernière version.
}
de \verb+tnsseq+ à destination de l'utilisateur principalement.
Tous les changements sont disponibles uniquement en anglais dans le dossier \verb+change-log+ : voir le code source de \verb+tnsseq+ sur \verb+github+.

\begin{description}
% Changes shown - START

    \medskip
    \item[2021-03-03] Nouvelle version mineure \verb+0.2.0-beta+.
    
    \begin{itemize}[itemsep=.5em]
        \item \topic*{Changements internes}
              les packages \verb#bm# et \verb#yhmath# étaient chargés à tort.   
    \end{itemize}
    
    \separation

% ------------------------ %

    \medskip
    \item[2020-08-08] Nouvelle version mineure \verb+0.1.0-beta+.
    
    \begin{itemize}[itemsep=.5em]
        \item \topic*{Comparaison asymptotique}
              ce sont de vrais opérateurs mathématiques qui sont définis en coulisse \emph{(du coup les macros \macro{bigO}, \macro{smallO}, \macro{bigOmega} et \macro{bigTheta} n'ont plus d'argument)}.   
    \end{itemize}
    
    \separation

% ------------------------ %

    \medskip
    \item[2020-07-10] Première version \verb+0.0.0-beta+.
% ------------------------ %

% Changes shown - END 
\end{description}


\newpage
\section{Toutes les fiches techniques} \label{techincal-ids}









\subsection{Des notations complémentaires pour des suites spéciales}



\IDmacro[a]{seqplus}{2}

\IDarg{1} l'exposant à droite.

\IDarg{2} l'indice à droite.


\separation


\IDmacro[a]{seqhypergeo}{2}

\IDarg{1} l'indice à gauche.

\IDarg{2} l'indice à droite.


\separation


\IDmacro[a]{seqsuprageo}{4}

\IDarg{1} l'indice à gauche.

\IDarg{2} l'indice à droite.

\IDarg{3} l'exposant à droite.

\IDarg{4} l'exposant à gauche.


\subsection{Sommes et produits en mode ligne}



Les opérateurs suivants ont un comportement spécifique vis à vis des mises en index et en exposant. 


\separation


\IDope{dprod}

\IDope{dsum }


\subsection{Comparaison asymptotique de suites et de fonctions}

\subsubsection{\texorpdfstring{Les notations $\bigO$ et $\smallO$}%
                           {Les notations "grand O" et "petit O"}}



\IDope{bigO}

\IDope{smallO}


% ---------------------- %


\subsubsection{\texorpdfstring{La notation $\bigomega$}%
                           {La notation "grand Omega"}}



\IDope{bigomega}


% ---------------------- %


\subsubsection{\texorpdfstring{La notation $\bigtheta$}%
                           {La notation "grand Theta"}}



\IDope{bigtheta}




\end{document}
