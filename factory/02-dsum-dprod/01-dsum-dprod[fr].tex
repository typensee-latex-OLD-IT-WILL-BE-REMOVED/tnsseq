\documentclass[12pt,a4paper]{article}

\makeatletter
	\input{../config/header[fr].sty}

	\usepackage{01-dsum-dprod}
\makeatother



\begin{document}

\section{Sommes et produits en mode ligne}

Pour limiter l'espace, \LaTeX{} affiche $\sum_{k=0}^{n}$ et non $\dsum_{k=0}^{n}$ sauf si l'on utilise la commande \macro{displaystyle}.
Les macros \macro{dsum} et \macro{dprod} permettent de se passer de \macro{displaystyle}.
Voici un exemple.


\begin{latexex}
 $\dsum_{k=0}^{n} 2^k
= \sum_{k=0}^{n} 2^k$

 $\dprod_{k=1}^{n} k
= \prod_{k=1}^{n} k$
\end{latexex}


\begin{remark}
	On peut taper  $\dsum_{k=0}^{n} \frac{1}{k}$ où la fraction n'est pas en mode \macro{displaystyle}.
\end{remark}


% ---------------------- %


\section{Fiches techniques}

Les opérateurs suivants ont un comportement spécifique vis à vis des mises en index et en exposant. 


\separation


\IDope{dprod}

\IDope{dsum }

\end{document}
